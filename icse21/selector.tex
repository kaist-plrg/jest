\subsection{Target Selector}

At the first step, \textsf{Target Selector} initializes the program pool with
the seed programs synthesized by \textsf{Seed Synthesizer}.  Then, it repeatedly
picks a target program that will be mutated in \textsf{Program Mutator}.  To
focus on increasing the semantics coverage, we use the branch coverage
of the current program pool when selecting the target program.

For example, Figure~\ref{fig:example-algo} is an excerpt of the \textbf{Abstract
Equality Comparison} algorithm that describes the semantics of non-strict
equality comparison operators (\code{==} or \code{\!=}).  Assume that the
following programs are the current program pool:
\begin{lstlisting}[style=myJSstyle]
1 + 2;  true == false;  0 == 1;
\end{lstlisting}
In this case, all pairs of left- and right-hand sides of the equality operator
in the current program pool have the same typed values.  Thus, in the first step
in the algorithm, the condition ``Type($x$) is the same as Type($y$)'' is always
true and the false branch is not covered.  To cover this false branch, the
\textsf{Target Selector} selects a program that passes the true branch such as
\code{true == false;} or \code{0 == 1;}.  If the program \code{true == false;} is
selected and mutated to \code{42 == false;} by \textsf{Program Mutator}, the
new program covers more semantics thus the program pool is extended as follows:
\begin{lstlisting}[style=myJSstyle]
1 + 2;  true == false;  0 == 1;  42 == false;
\end{lstlisting}
Then, \textsf{Target Selector} has more choices of branches in steps 2, 3, 10,
and 11 to select next target program.  When it targets the branch in the step
10, it selects the code \code{42 == false;} as the target program.  In the same
way, \textsf{Target Selector} iterates this process until the semantics coverage
is converged.
